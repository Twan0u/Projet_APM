\documentclass[a4paper,11pt]{article}
\usepackage[T1]{fontenc}
\usepackage[utf8]{inputenc}
\usepackage{lmodern}

\title{PreDossier APM}
\author{Nathan Surquin\\
William Michaux\\
Antoine Lambert}

\begin{document}

\maketitle
\newpage
\tableofcontents
\newpage
\section{Ennoncé}
\subsection{Gestion du stock}
Selon différent facteurs, le gestionnaire de stock commande des marchandises aux fournisseurs.
\subsubsection{Objectif du gestionnaire de stock}
Déterminer la quantité de stock nécessaire : quantité minimum et de réassort de chaque     produit. Tenir compte de : commandes effectives, délais de livraison et évènements à venir. A une date donnée (date souhaitée), le manager de stock (=gestionnaire de stock) calcule l’image de stock et génère la commande de réassort. Cette commande doit être validée par le manager (gérant de l’entreprise, !=gestionnaire de stock).
\subsubsection{Retour vidange}
Le gestionnaire de stock (gère aussi le retour des vidanges au fournisseur) génère un bon de retour à destination du fournisseur. A chaque retour de vidange fournisseur, un bon de remboursement est créé pour la comptabilité (de la brasserie), service qui pourra ensuite valider le remboursement du fournisseur.
\subsubsection{Réception de la marchandise}
Le gestionnaire de stock vérifie chaque livraison (en termes de quantité et de conformité). Ensuite, il entre les marchandises dans le stock et crée une facture pour la comptabilité. Ce service utilisera cette facture reçue du gestionnaire de stock pour vérifier la facture ultérieure du fournisseur. (‘celle créée par le gestionnaire de stock sert donc de sécurité pour vérifier qu’ils ne seront pas surtaxés’). (le gestionnaire de stock est responsable des articles dans le stock et de leur structure : la brasserie ne gère officiellement pas de « vrac » mais certains produits sont séparés de leur pack exemple : un casier de bière entre mais un seule sort).
\subsection{Gestion des ventes comptoir}
Les clients passent au drive-in.Ils peuvent payer par virement ultérieur ou en cash. Le gestionnaire de client décide du statut du client et de la limite de crédit accordée.
\subsection{Gesion des commandes}
Les clients peuvent passer commande pour une date précise. Le client peut payer une partie ou la totalité mais il doit payer un acompte sauf exception
Si livraison, un prix est calculé sur base de zone, certains clients ont une réduction.
Réduction globale possible en fonction du grade du client.
Lorsqu’un acompte est nécessaire, la préparation de la commande ne commence que quand l’acompte est payé, sinon elle passe direct a la préparation.
La veille de la livraison, le préparateur de commande prépare la commande et la charge dans un camion. Si un produit est manquant (entièrement ou en partie), le préparateur indique la vraie quantité à coté de la quantité commandée. Le matin, le livreur livre et valide la commande, et met sur le bon de livraison : les marchandises et la quantité de chaque produit déposé.
Une fois la tournée finie, le bon de livraison est transmis au gestionnaire de stock qui adapte les quantités livrées sur le bon de commande. Car c’est le bon de commande qui est utilisé par le comptable pour émettre la facture au client.
Si le client ne paie pas dans les deux semaines, un rappel est envoyé. Après deux rappels, la facture est transmise à une société de recouvrement et n’est donc plus en charge du comptable.

\subsection{Gestion des retour}
Les marchandises non entamées peuvent être rapportées et sont remises au prix de vente. Une note de crédit est créée, elle est soit remboursée en liquide soit par compte bancaire. Si par compte bancaire, la note de crédit n’est pas acquittée par le client et la compta fait le remboursement. Le système vérifie alors si on le num du compte du client dans sa fiche.
\end{document}
